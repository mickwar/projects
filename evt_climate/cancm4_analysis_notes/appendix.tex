\appendix
\section{Appendix}

\subsection{Likelihood for hierarchical model}

\begin{align*}
L(\m{y}; \m{\sigma}, \m{\xi}, \m{\zeta}) &= \prod_{i=1}^R f_{Y_i}(\m{y}_i|\sigma_i,\xi_i,\zeta_i) \nonumber \\
&= \prod_{i=1}^R\left[\prod_{j\in A_i} F_{X_i}(u) \times \prod_{j\in A_i^c} f_{X_i}(y_{ij}+u)\right] \nonumber \\
&\approx \prod_{i=1}^R\left[\prod_{j\in A_i} F_{X_i}(u) \times \prod_{j\in A_i^c} [1-F_{X_i}(u)]h(y_{ij}|\sigma_i,\xi_i)\right]~~~~~\mathrm{(approximation~(\ref{gpapprox}))} \nonumber \\
&= \prod_{i=1}^R\left[\prod_{j\in A_i} (1-\zeta_i)\times \prod_{j\in A_i^c} \frac{\zeta_i}{\sigma_i}\left(1+\xi_i\frac{y_{ij}}{\sigma_i}\right)_+^{-1/\xi_i-1}\right]~~~~~(\zeta_i=1-F_{X_i}(u)) \nonumber \\
&= \prod_{i=1}^R\left[(1-\zeta_i)^{n_i-k_i}\zeta_i^{k_i}\prod_{j\in A_i^c}\frac{1}{\sigma_i}\left(1+\xi_i\frac{y_{ij}}{\sigma_i}\right)_+^{-1/\xi_i-1}\right]
\end{align*}

\subsection{Definition of a simple Pareto process}
\label{def_spp}

The constructive definition of a simple Pareto process is as follows (from Theorem 2.1 of \cite{ferreira2014generalized}):

Let $C^+(S)$ be the space of non-negative real continuous functions on S, with S some compact subset of $\R^d$4. Let $W$ be a stochastic process in $C^+(S)$ and $\omega_0$ a positive constant. When $W$ satisfies:
\begin{enumerate}[(a)]
\item $V\in C^+(S)$ is a stochastic process satisfying $\sup_{s\in S} V(s)=\omega_0$ almost surely, and $E[V(s)] > 0$ for all $s\in S$.
\item $Y$ is a standard Pareto random variable, $P(Y\leq y)=1-1/y$, $y>1$,
\item $Y$ and $V$ are independent.
\end{enumerate}
then $W$ is called a simple Pareto process.

\subsection{Measure of asymptotic dependence for simple Pareto process}

Provided $V_1>0$ and $V_2>0$,
\begin{align*}
\chi &= \lim_{u\rightarrow 1} P(F_{W_1}(W_1) > u | F_{W_2}(W_2) > u) \\
&= \lim_{u\rightarrow 1} P\left(1-\frac{E(V_1)}{W_1} > u \middle| 1-\frac{E(V_2)}{W_2} > u\right) \\
&= \lim_{u\rightarrow 1} P\left(W_1 > \frac{E(V_1)}{1-u} \middle| W_2 > \frac{E(V_2)}{1-u}\right) \\
&= \lim_{u\rightarrow 1} \frac{P\left(W_1 > \frac{E(V_1)}{1-u}, W_2 > \frac{E(V_2)}{1-u}\right)}{P\left(W_2 > \frac{E(V_2)}{1-u}\right)} \\
&= \lim_{u\rightarrow 1} \frac{P\left(YV_1 > \frac{E(V_1)}{1-u}, YV_2 > \frac{E(V_2)}{1-u}\right)}{1-u} \\
&= \lim_{u\rightarrow 1} \frac{1}{1-u}P\left(Y > \frac{E(V_1)}{(1-u)V_1}, Y > \frac{E(V_2)}{(1-u)V_2}\right) \\
&= \lim_{u\rightarrow 1} \frac{1}{1-u}P\left(Y > \frac{E(V_1)}{(1-u)V_1} \vee \frac{E(V_2)}{(1-u)V_2}\right) \\
&= \lim_{u\rightarrow 1} \frac{1}{1-u}P\left(Y > \frac{1}{1-u}\left(\frac{E(V_1)}{V_1} \vee \frac{E(V_2)}{V_2}\right)\right) \\
&= \lim_{u\rightarrow 1} \frac{1}{1-u}(1-u) E\left[\left(\frac{E(V_1)}{V_1} \vee \frac{E(V_2)}{V_2}\right)^{-1}\right] \\
&= E\left(\frac{V_1}{E(V_1)} \wedge \frac{V_2}{E(V_2)}\right) \\
\end{align*}


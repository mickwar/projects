\begin{Large}
% \noindent \textbf{Extreme value comparison of CanCM4 simulations and observations}
% \noindent \textbf{Comparison of extreme values of different climate model simulations and observations}
%\noindent \textbf{Extreme value comparison of different climate model simulations and observations}
\noindent \textbf{Extreme value comparison of climate model simulations and observations}
\end{Large}
\bigskip

%\begin{large}
\noindent Mickey Warner, Bruno Sans{\'o}
%end{large}


\bigskip
\bigskip
\begin{quote}
\textbf{Abstract.}
Climate models predict the intensity of extreme weather patterns, thus it is important to assess how similar the extreme behavior in climate model simulations is to that of observations.
%We explore similarities and differences in extreme behavior between climate simulations and observational records.
We fit a Bayesian hierarchical threshold exceedance model to simulations from the climate model CanCM4.
Three simulation classes are analyzed and compared---decadal, historical, and pre-industrial control---as well as an observation product. 
%(Extremal data over California and the contiguous United States.)
%Three simulation classes are analyzed and compared: decadal, historical, and pre-industrial control.
%We extend the comparison to an observation product.
To assess the extremes of the series considered we fit a generalized Pareto model to the exceedances over a threshold.
%This model is applied to data over California and the contiguous United States.
Our method includes a likelihood-based hierarchical model for declustering.
Comparisons for relevant distributions that result from the analysis are made visually with posterior parameter intervals and numerically using the Bhattacharyya distance between probability densities.
We find that in some domains, the simulations are in agreement among themselves and with the observations, but in others they are quite different.
In order to study the joint tail behavior of simulations and observations, we perform a bivariate extreme value analysis using simple Pareto processes in conjunction with a Bayesian non-parametric model of an angular measure.
The results show weak to moderate tail dependence in nearly every comparison made.
\end{quote}


\section{Introduction}
\label{intro}


Climate models predict an intensification of extreme activity in weather patterns \citep{easterling2000climate}. Extreme weather has social, ecological, and political impacts, necessitating the need to gauge the differences and similarities in extreme behavior between climate simulations and observations.

The fifth phase of the Coupled Model Intercomparison Project (CMIP5) brings together a variety of climate models under an experimental design framework \citep{taylor2012overview,hibbard2007strategy}. Some of the stated goals of CMIP5 is to examine the variability and predictability of climate models and to explore why some models differ in their responses. The experimental design of CMIP5 is made up of both long-term and short-term simulations. The long-term simulations are run far into the future while the short-term simulations, called decadal, are typically executed for $10$- and $30$-year forecasts. These decadal runs aim at improving the prediction skill of climate models through time-evolving regional climate conditions and external forcings \citep{meehl2009decadal}. The innovative experimental design of CMIP5 allows for assessing the prediction skill and internal variability of these decadal runs \citep{kim2012evaluation}. Should decadal simulations come to the forefront of climate model predictions, it becomes necessary to look at all predictive aspects of this simulation class, including, as discussed in this paper, the extremal behavior. Some recent work on extremes in climate simulations include \cite{weller2012investigation}, \cite{weller2013two}, and \cite{fix2016comparison}.

For our analysis we consider a specific climate model, CanCM4 (described in more detail in section \ref{clim_sim}). This model includes decadal, historical, and pre-industrial control runs for about a 50 year period. Because of this, CanCM4 is a viable option for modeling and comparing the extremes, but our method could just as well be applied to other climate models.

In this paper we address at least two questions: 1) Does using the same climate model in different ways produce similar extreme behavior? and 2) How well do the simulations agree with the observations? Answers to these questions would inform us whether the climate model is a reasonable representation of observed extremes.

% Bruno: The issue of replication is particular to this problem. You should talk about it because I don't think this is something that is commonly considered in extreme value analysis.

The classic approach to analyzing extreme values is to model block maxima (e.g. annual maxima). It can be shown that under certain conditions the block maxima of independent random variables have a distribution which belongs to the generalized extreme value (GEV) family of distributions \citep{coles2001introduction}. Such an approach naturally requires omitting a large portion of the data. This can be remedied by using a threshold exceedance model which involves selecting some large threshold and fitting the exceedances to the generalized Pareto distribution (GPD).

% Reference for peaks-over-threshold?

CanCM4 climate simulations are run at several different initial conditions producing multiple replicates. Utilizing replicates is not standard in extreme value analysis, but is very naturally handled with Bayesian hierarchical models. This includes not only models for the exceedances, but also a hierarchical formulation for estimating the extremal index, a parameter which describes cluster sizes of exceedances in the limit. Such models allow us to make inference on the underlying climate model process and it is this process that is to be compared with the observations.

A univariate analysis would be sufficient if we desire a marginal comparison only. To understand how several random variables interact in the extremes, we need a multivariate extremal model. A climate simulation may act similarly to observations in the margin, but their respective extreme values may occur at drastically different times. A multivariate extreme value model allows us explore tail dependence. For our multivariate model, we use simple Pareto processes \citep{ferreira2014generalized} which involves the estimation of an angular measure. A non-parametric method for multivariate extremes is found in \cite{goix2015sparsity}.

% Bring other references from section 4 here?


The remainder of the paper is outlined as follows. In section \ref{process} we describe the data in more detail as well as the adjustments made to make the simulations comparable to the observations. The univariate model, including the hierarchical models for the exceedances and the extremal index, and quantities useful for comparison are given in section \ref{margin}. Section \ref{bivariate} describes our bivariate approach. Results are presented in section \ref{results} and we conclude with a discussion in section \ref{conclusions}.






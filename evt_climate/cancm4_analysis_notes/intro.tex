\begin{Large}
% \noindent \textbf{Extreme value comparison of CanCM4 simulations and observations}
% \noindent \textbf{Comparison of extreme values of different climate model simulations and observations}
\noindent \textbf{Extreme value comparison of different climate model simulations and observations}
\end{Large}
\bigskip

%\begin{large}
\noindent Mickey Warner, Bruno Sans{\'o}
%end{large}


\bigskip
\bigskip
\begin{quote}
\textbf{Abstract.}
Climate models predict the intensity of extreme weather patterns, thus it is important to assess how similar the extreme behavior in climate model simulations is to that of observations.
%We explore similarities and differences in extreme behavior between climate simulations and observational records.
We fit a Bayesian hierarchical threshold exceedance model to simulations from the climate model CanCM4.
Three simulation classes are analyzed and compared: decadal, historical, and pre-industrial control.
We extend the comparison to an observation product.
To assess the extremes of the series considered we fit a generalized Pareto model to the exceedances over a threshold.
This model is applied to data over California and the contiguous United States.
Comparisons for relevant distributions that result from the analysis are made visually with posterior parameter intervals and numerically using the Bhattacharyya distance between probability densities.
We find that in some domains, the simulations are in agreement among themselves and with the observations, but in others they are quite different.
In order to study the joint tail behavior of simulations and observations, we perform a bivariate extreme value analysis using simple Pareto processes in conjunction with a Bayesian non-parametric model. The results show weak to moderate tail dependence in nearly every comparison made [which are simulations to observations].
\end{quote}
%[You should say something about the methodological innovations of your analysis: hierarchical formulation; likelihood-based hierarchical model for declustering; extensions to a flexible model-based approach for bivariate extremes;]


\section{Introduction}
\label{intro}

The main focus of this paper is to compare the extreme values of an observation product with those of CanCM4 climate simulations. We address at least two questions: 1) Does using the same climate model in different ways produce similar extreme behavior? and 2) How well do the simulations agree with the observations? Answers to these questions would inform us whether the climate model is a reasonable representation of observed extremes.

The classic approach to analyzing extreme values is to model block maxima (e.g. annual maxima). It can be shown that under certain conditions the block maxima of independent random variables have a distribution which belongs to the generalized extreme value (GEV) family of distributions \citep{coles2001introduction}. Such an approach naturally requires omitting a large portion of the data. This can be remedied by using a threshold exceedance model which involves selecting some large threshold and fitting the exceedances to the generalized Pareto distribution (GPD). See \cite{coles2001introduction} for an excellent introduction to these and other approaches in extreme value analysis.

% \cite{weller2013two}    % for modeling extremes


% Being a threshold exceedance analysis, we must concern ourselves with exceedances occuring together within a short time. This is handled by studying the extremal index $\theta$, a measure of dependence among the extremes. With an estimate for $\theta$, we can ``decluster'' the exceedances to obtain independent clusters. The method for estimating $\theta$ and declustering has been generalized to the hierarchical setting, see section \ref{index}.

% Having replicates of a time-series suggests the use of a hierarchical model, described in detail in section \ref{hier}. Under such a framework we can model each series separately, while assuming these series come from a larger population. In the analysis, we will place focus on the mean of this larger population, being akin to the ensemble average in a climate study.

We take a Bayesian approach to modeling threshold exceedances (section \ref{thresh}). Under this framework, taking into account the replicated experiments is naturally addressed with a hierarchical model.

We attempt to answer the questions posed earlier by comparing posterior intervals for statistical model parameters and other quantities such as return level (section \ref{return}) and Bhattacharyya distance (section \ref{bhatta}). Additionally, the bivariate analysis (section \ref{bivariate}) allows us to measure the strength of tail dependence between simulations and observations.



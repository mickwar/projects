\documentclass[12pt]{article}

\usepackage{natbib}
\usepackage{amssymb}
\usepackage{amsmath}
\usepackage{bm}
\usepackage{dsfont}
\usepackage[margin=1.00in]{geometry}
\usepackage[font=scriptsize]{caption}
\usepackage{dsfont}
\usepackage{amsmath}
\usepackage{graphicx}
\usepackage{bm}
\usepackage{enumerate}
\usepackage[shortlabels]{enumitem}
\newcommand{\m}[1]{\mathbf{\bm{#1}}}
\newcommand{\R}{I\hspace{-4.4pt}R}
\newcommand{\bc}[1]{\textcolor{blue}{\mathbf{#1}}}
\newcommand{\ind}{\mathds{1}}

\setlength{\parindent}{0pt}

\begin{document}

From the simple Pareto process
\begin{align}
\chi(q) &= Pr(F_{W_1}(W_1) > q | F_{W_2}(W_2) > q)  \\
        &= E\left(\frac{V_1}{E(V_1)} \wedge \frac{V_2}{E(V_2)}\right).
\end{align}
So $\chi(q)$ is the same value for all $0\leq q\leq 1$ and this is what is reported as the measure of asymptotic dependence. Converting from the Pareto process $W$ back to the original data $X$ using
\begin{align}
W_i = YV_i = TX_i = \left(1 + \xi_i\frac{X_i - u_i}{\sigma_i}\right)_+^{1/\xi_i}
\end{align}
we can compute a correspondence between two variables $X_1$ and $X_2$ (which could represent, say, observations and climate simulations) by
\begin{align}
\chi(q) &= P\left(W_1 > \frac{E(V_1)}{1-u} \middle| W_2 > \frac{E(V_2)}{1-u}\right) \\
    &= P\left(\left(1 + \xi_1\frac{X_1-u_1}{\sigma_1}\right)_+^{1/\xi_1} > \frac{E(V_1)}{1-q} \middle| \left(1 + \xi_2\frac{X_2-u_2}{\sigma_2}\right)_+^{1/\xi_2} > \frac{E(V_2)}{1-q}\right) \\
    &= P\left(X_1 > u_1 + \frac{\sigma_1}{\xi_1}\left[\left(\frac{E(V_1)}{1-q}\right)^{\xi_1}-1\right] \middle| X_2 > u_2 + \frac{\sigma_2}{\xi_2}\left[\left(\frac{E(V_2)}{1-q}\right)^{\xi_2}-1\right]\right) \\
    &= P\left(X_1 > x_1 \middle| X_2 > x_2)
\end{align}
From (2), it must be the case that (7) is the same for all $q$. I think this is the reason for the issue I'm having. Choosing any $x_1$ or $x_2$ in which we might be interested doesn't yield itself to a straightfoward calculation. There is a single probability $\chi(q)$ that corresponds to many pairs of $(x_1, x_2)$, which are both functions of $q$. And it would seem possible to find these pairs, but they are all share the same probability.
\bigskip

Ideally, we'd like to choose $x_1$ and $x_2$ to be very high, something like the thresholds. But even in this case, it doesn't appear that we can use the simple Pareto process calculation of (2). For $x_1$ and $x_2$ to be the thresholds only, we must have the additional terms in (6) be zero, but this cannot happen unless $E(V_1)=E(V_2)$, which cannot be true except in degenerate cases (if it were true, we would pick $1-q=E(V_1)=E(V_2)$).
\bigskip

With what we've done with the simple Pareto process, it is not clear to me how to come up with a distribution for (7), whether to get probabilities or return levels.

\end{document}

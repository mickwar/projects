\section{Hierarchical model}
\label{model}

Suppose we have $R$ simulations from a climate model. If we assume these simulations are indepedent from each other, then we expect there to be $R$ unique extremal indices $\theta_1,\ldots,\theta_R$. However, since these all come from the same climate model, we may wish to assume that the $\theta_i$ come from a common distribution
\[ \theta_i \overset{iid}\sim Beta\left(\theta\nu, (1-\theta)\nu\right). \]
Under model (\ref{ferro}), we place a similar prior on the $p_i$
\[ p_i \overset{iid}\sim Beta\left(p\tau, (1-p)\tau\right). \]
Since model (\ref{suveges}) does not properly allow for the estimation of $p_i$, we fix each $p_i$ to be the empirical estimate of $F_i(u)$.

The model is completed by choosing priors for $\theta$, $\nu$, $p$, and $\tau$---the latter two parameters being required only for model (\ref{ferro}). We assume
\begin{align*}
\theta &\sim Beta(a_\theta, b_\theta) \\
\nu &\sim Gamma(a_\nu, b_\nu) \\
p &\sim Beta(a_p, b_p) \\
\tau &\sim Gamma(a_\tau, b_\tau) 
\end{align*}
with the hyperparameters assumed to be
%\begin{table}[h]
\begin{center}
\begin{tabular}{rlcl}
$\theta$: & $a_\theta = 1/2        $ &~~& $b_\theta = 1               $ \\
$   \nu$: & $   a_\nu = 1          $ &~~& $   b_\nu = 1/10            $ \\
$     p$: & $     a_p = 100 \hat{F}$ &~~& $     b_p = 100 (1-\hat{F}) $ \\
$  \tau$: & $  a_\tau = 1          $ &~~& $  b_\tau = 1/10            $ \\
\end{tabular}
\end{center}
%\end{table}

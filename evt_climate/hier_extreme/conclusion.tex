\section{Conclusion}
\label{conclusion}

The simulation study seems to favor the intervals estimator (\ref{ferro}) over the maximum likelihood estimator of \cite{suveges2010model} for lower thresholds. This is problematic for the obvious reasons surrounding too-low thresholds. But when working with small amounts of data, a lower threshold may be the only viable option.

Analysis of the CanCM4 simulations showed differences between the likelihoods in the estimates of the extremal index. There is also a trend for $\theta$ to increase with threshold. This could suggest at least two things. First, that we need to pick a threshold sufficiently high before convergence to the true extremal index. Second, and possibly worse, that we are seeing the issues described in \cite{ferro2003inference} regarding model (\ref{ferro}).

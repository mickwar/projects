\documentclass[12pt]{article}

\usepackage{amssymb}
\usepackage{amsmath}
\usepackage{bm}
\usepackage{dsfont}
\usepackage[margin=1in]{geometry}
\usepackage[font=scriptsize]{caption}
/home/mickey/files/repos/latex/setup.tex
\newcommand{\bc}[1]{\textcolor{blue}{\mathbf{#1}}}
\newcommand{\ind}{\mathds{1}}
\newcommand{\m}[1]{\mathbf{\bm{#1}}}
\newcommand{\R}{I\hspace{-4.4pt}R}

\setlength\parindent{0pt}


\begin{document}

\section{Introduction}

We propose a hierarchical extension to standard univariate extreme value modeling. In the univariate setting it is common to work with a single time-series or spatial field. However, having multiple realizations from computer simulations at a variety of input settings suggests an extension to a hierarchical formulation.

\section{Threshold exceedance model}

A threshold exceedance model considers observations that are greater than some large threshold $u$. The standard approach is to use the approximation to the generalized Pareto distribution for the exceedances (see Theorem 4.1 of Coles (2001) page 75).
\bigskip

Let $X_1,X_2,\ldots$ be a sequence of independent random variables with common distribution. Then, assuming the conditions of the theorem hold, for large enough $u$, the distribution of $(X-u)$, conditional on $X>u$ is approximately
\begin{align}
P(X-u\leq y|X>u) \approx H(y) = 1 - \left(1+\frac{\xi y}{\sigma}\right)^{-1/\xi} \label{gpapprox}
\end{align}
defined on $\{y:y>0~\mathrm{and}~(1+\xi y/\sigma) >0\}$. $H(y)$ is the distribution function for a generalized Pareto random variable with parameters $\sigma>0$ and $\xi$.
\bigskip

For excesses $y_1,\ldots,y_k$ of a threshold $u$, the likelihood of $(\sigma,\xi)$ is derived from (\ref{gpapprox}) as
\begin{align}
L(y_1,\ldots,y_k;\sigma,\xi)=\sigma^{-k}\sum_{i=1}^k\left(1+\frac{\xi y_i}{\sigma}\right)_+^{-1/\xi-1}
\end{align}
where $z_+=\max(z,0)$.
\bigskip

% \begin{align*}
% P(X-u\leq y|X > u) &= P(X\leq y+u|X > u) \\
%  &= 1 - P(X> y+u|X > u) \\
%  &= 1 - \frac{P(X>y+u, X>u)}{P(X>u)}  \\
%  &= 1 - \frac{P(X>y+u)}{P(X>u)}  \\
%  &= 1 - \frac{1-P(X-u \leq y)}{P(X>u)}  \\
% \Rightarrow P(Y\leq y) = P(X-u\leq y) &= 1-P(X>u)\left(1 - P(X-u\leq y|X>u)\right) \\
%  &= 1-\zeta\left(1+\frac{\xi y}{\sigma}\right)^{-1/xi} \\
% \end{align*}
% where $\zeta=P(X>u)$.
% \bigskip

\subsection{Hierarchical model}

To extend this to the hierarchical setting, suppose we have $R$ replicates or computer simulations, each with $n_i$ observations, for $i=1,\ldots,R$. Let $X_{ij}$ denote the $j$th observation in replicate $i$. We assume
\[ X_{ij} \overset{ind}\sim F_i,~~~~~i=1,\ldots,R,~~~~~j=1,\ldots,n_i. \]

For a fixed $u$ and each $i$, define the following sets:
\[ A_i = \{j:x_{ij}\leq u\},~~~ A_i^c = \{j: x_{ij}>u\} \]
where $|A_i|=n_i-k_i$ and $|A_i^c|=k_i$ with $k_i$ being the number of exceedances in replicate $i$. We define our exceedances as
\[ y_{ij} = (x_{ij}-u)\cdot \ind_{(j \in A_i^c)} \]
so that all observations not exceeding $u$ are marked as $0$. Let $\m{y}_i=(y_{i,1},\ldots,y_{i,n_i})^\top$ and $\m{y}=(\m{y}_1^\top,\ldots,\m{y}_R^\top)^\top$.
\bigskip

\noindent The likelihood is given by
\begin{eqnarray*}
L(\m{y}; \m{\sigma}, \m{\xi}, \m{\zeta}) &=& \prod_{i=1}^R f_{Y_i}(\m{y}_i|\sigma_i,\xi_i,\zeta_i) \\
&=& \prod_{i=1}^R\left[\prod_{j\in A_i} F_{X_i}(u) \times \prod_{j\in A_i^c} f_{X_i}(y_{ij}+u)\right] \\
&\approx& \prod_{i=1}^R\left[\prod_{j\in A_i} F_{X_i}(u) \times \prod_{j\in A_i^c} [1-F_{X_i}(u)]h(y_{ij}|\sigma_i,\xi_i)\right] \\
&=& \prod_{i=1}^R\left[\prod_{j\in A_i} (1-\zeta_i)\times \prod_{j\in A_i^c} \frac{\zeta_i}{\sigma_i}\left(1+\xi_i\frac{y_{ij}}{\sigma_i}\right)_+^{-1/\xi_i-1}\right] \\
\end{eqnarray*}

\noindent We are left with
\[ L(\m{y}; \m{\sigma}, \m{\xi}, \m{\zeta}) = \prod_{i=1}^R\left[(1-\zeta_i)^{n_i-k_i}\zeta_i^{k_i}\prod_{j\in A_i^c}\frac{1}{\sigma_i}\left(1+\xi_i\frac{y_{ij}}{\sigma_i}\right)_+^{-1/\xi_i-1}\right] \]
\noindent Note that the parameters describing the tail of $F_i$ (i.e. $\sigma_i,\xi_i$) depend only on those observations which exceeded $u$.



\noindent These priors complete the hierarchical model formulation. Greek letters are random variables while English letters are fixed.
\begin{eqnarray*}
\sigma_i|\alpha, \beta &\sim& Gamma(\alpha, \beta) \\
\xi_i|\xi, \tau^2  &\sim& Normal(\xi, \tau^2) \\
\zeta_i|\mu, \eta &\sim& Beta(\mu\eta, (1-\mu)\eta) \\
%\theta_i|\theta_\mu, \theta_\tau &\sim& Beta(\theta_\mu\theta_\tau, (1-\theta_\mu)\theta_\tau) \\
 \\
\alpha_\sigma \sim Gamma(a_\alpha, b_\alpha)&  &\beta_\sigma \sim Gamma(a_\beta, b_\beta) \\
\xi \sim Normal(m, s^2)&  &\tau^2 \sim Gamma(a_\tau, b_\tau) \\
\mu \sim Beta(a_\mu, b_\mu)&  &\eta \sim Gamma(a_\eta, b_\eta) \\
%\theta_\mu \sim Beta(a_{\theta_\mu}, b_{\theta_\mu})&  &\theta_\tau \sim Gamma(a_{\theta_\tau}, b_{\theta_\tau})
\end{eqnarray*}








\section{De-trending}



\section{Return levels}



\section{Extremal Index}

\textbf{Theorem.} (Coles 2001, p. 96) Let $X_1,X_2,\ldots$ be a stationary process and $X_1^*,X_2^*,\ldots$ be a sequence of independent variables with the same marginal distribution. Define $M_n=\max\{X_1,\ldots,X_n\}$ and $M_n^*=\{X_1^*,\ldots,X_n^*\}$. Under suitable regularity conditions,
\[ Pr\{(M_n^*-b_n)/a_n\leq z\} \rightarrow G_1(z) \]
as $n\rightarrow\infty$ for normalizing sequences $\{a_n > 0\}$ and $\{b_n\}$, where $G_1$ is a non-degenerate distribution function, if and only if
\[ Pr\{(M_n-b_n)/a_n\leq z\} \rightarrow G_2(z), \]
where
\[ G_2(z)=G_1^\theta(z) \]
for a constant $\theta$ such that $0<\theta\leq 1$. \hfill $\square$
\bigskip

$\theta$ is called the extremal index and has the following (loose) interpretation
\[ \theta = (\mathrm{limiting~mean~cluster~size})^{-1}, \]
where limiting is in the sens of clusters of exceedances of increasingly high thresholds.
\bigskip

For a given threshold $u$, let $1\leq E_1 < \cdots < E_N \leq n$ be the exceedance times. That is, for $n$ observations, $N$ of them exceed $u$ and the time at which the exceedance occurs as given by the $E_i$. The observed interexceedance times are $T_i=E_{i+1}-E_i$, for $i=1,\ldots,N-1$.
\bigskip

Ferro and Segers (2003) provide the following estimator for $\theta$
\[ \widetilde{\theta}=\begin{cases} \min(1, \tilde{\theta}_1) & \mathrm{~~~~~if~} \max\{T_i: 1\leq i \leq N-1\} \leq 2  \\ \min(1, \tilde{\theta}_2) & \mathrm{~~~~~if~} \max\{T_i:1\leq i \leq N-1\} > 2 \end{cases} \]
where
\[ \tilde{\theta}_1 = \frac{2\left(\sum_{i-1}^{N-1}T_i\right)^2}{(N-1)\sum_{i=1}^{N-1}T_i^2} \]
and
\[ \tilde{\theta}_2 = \frac{2\left[\sum_{i-1}^{N-1}(T_i-1)\right]^2}{(N-1)\sum_{i=1}^{N-1}(T_i-1)(T_i-2)}. \]
\bigskip

If $\max{T_i}\leq 2$, then $\tilde{\theta}_1$ is used as the estimator for $\theta$, but it can be shown that in this case $\tilde{\theta}_1$ always evaluates to a number greater than unity. So $\widetilde{\theta}$ would always evaluate to 1. This can be a problem when working with smaller datasets.
\bigskip

Ferro and Segers also provide the following likelihood
\[ L_1(\theta, p) = (1-\theta p^\theta)^{m_1}[\theta(1-p^\theta)]^{N-1-m_1}p^{\theta\sum_{i=1}^{N-1}(T_i-1)} \]
where $m_1=\sum_{i=1}^{N-1}I(T_i=1)$ and $p=F(u)=1-\bar{F}(u)$.
\bigskip

S{\"u}veges (2007) derives an estimator based on the transformation $S_i=T_i-1$,
\[ \hat{\theta} = \frac{ \sum_{i=1}^{N-1}qS_i +N-1-N_C-\left[\left(\sum_{i=1}^{N-1}qS_i+N-1+N_C\right)^2-8N_C\sum_{i=1}^{N-1}qS_i\right]^{1/2}}{2\sum_{i=1}^{N-1}qS_i} \]
where $N_C=\sum_{i=1}^{N-1}I(S_i\neq 0)=\sum_{i=1}^{N-1}I(T_i \neq 1)=N-1-m_1$ and $q=1-p$. Her estimator is the maximum likelihood estimator for the likelihood based on $S_i$,
\[ L_2(\theta, q)= (1-\theta)^{N-1-N_C}\theta^{2N_C}e^{-\theta q \sum_{i=1}^{N-1}S_i}. \]
\bigskip


\section{Hierarchial formulation}





\subsection*{Side notes}

Units?
\bigskip

Visualizing the analysis on several domains?
\bigskip

Extremal: need to enforce $r = \theta p^\theta \leq p^\theta = q$. Use prior $p(r,q)=p(r|q)p(q)$ where $p(r|q)$ is a truncated beta on $0\leq r \leq q$, and $p(q)$ is beta.
\bigskip

Extremal: how to handle interexceedance times on groups? Suppose we have a time series that is 200 observations long. Due to some seasonal effects, we only wish to examine Obs 1--50 and Obs 101--150. If we observe an exceedance at Obs 45 and Obs 107, do we compute an interexceedance time as $107-45=62$? Do we consider a ``new'' data set that discards Obs 51--100 and 151--200, and so compute the interexceedance time as $57-45=12$ where the new Obs 57 is the old Obs 107? Or do we simply ignore the interexceedance time between the groups? How does this affect our estimate of the extremal index?


\subsection*{Extremal index simulation study}

\subsubsection*{Hierarchical}

\newpage

\begin{figure}
\begin{center}
\includegraphics[width=5.5in, height=2.45in]{../extremal_comparison/figs/sim_frechet_hier_15_250_5.pdf}
\caption{$\theta=0.15$, $n=250$, $R=5$}
\end{center}
\end{figure}

\begin{figure}
\begin{center}
\includegraphics[width=5.5in, height=2.45in]{../extremal_comparison/figs/sim_frechet_hier_15_500_5.pdf}
\caption{$\theta=0.15$, $n=500$, $R=5$}
\end{center}
\end{figure}

\begin{figure}
\begin{center}
\includegraphics[width=5.5in, height=2.45in]{../extremal_comparison/figs/sim_frechet_hier_15_1000_5.pdf}
\caption{$\theta=0.15$, $n=1000$, $R=5$}
\end{center}
\end{figure}

\newpage

\begin{figure}
\begin{center}
\includegraphics[width=5.5in, height=2.45in]{../extremal_comparison/figs/sim_frechet_hier_15_250_10.pdf}
\caption{$\theta=0.15$, $n=250$, $R=10$}
\end{center}
\end{figure}

\begin{figure}
\begin{center}
\includegraphics[width=5.5in, height=2.45in]{../extremal_comparison/figs/sim_frechet_hier_15_500_10.pdf}
\caption{$\theta=0.15$, $n=500$, $R=10$}
\end{center}
\end{figure}

\begin{figure}
\begin{center}
\includegraphics[width=5.5in, height=2.45in]{../extremal_comparison/figs/sim_frechet_hier_15_1000_10.pdf}
\caption{$\theta=0.15$, $n=1000$, $R=10$}
\end{center}
\end{figure}

\newpage

\begin{figure}
\begin{center}
\includegraphics[width=5.5in, height=2.45in]{../extremal_comparison/figs/sim_frechet_hier_15_250_20.pdf}
\caption{$\theta=0.15$, $n=250$, $R=20$}
\end{center}
\end{figure}

\begin{figure}
\begin{center}
\includegraphics[width=5.5in, height=2.45in]{../extremal_comparison/figs/sim_frechet_hier_15_500_20.pdf}
\caption{$\theta=0.15$, $n=500$, $R=20$}
\end{center}
\end{figure}

\begin{figure}
\begin{center}
\includegraphics[width=5.5in, height=2.45in]{../extremal_comparison/figs/sim_frechet_hier_15_1000_20.pdf}
\caption{$\theta=0.15$, $n=1000$, $R=20$}
\end{center}
\end{figure}







\newpage

\begin{figure}
\begin{center}
\includegraphics[width=5.5in, height=2.45in]{../extremal_comparison/figs/sim_frechet_hier_50_250_5.pdf}
\caption{$\theta=0.50$, $n=250$, $R=5$}
\end{center}
\end{figure}

\begin{figure}
\begin{center}
\includegraphics[width=5.5in, height=2.45in]{../extremal_comparison/figs/sim_frechet_hier_50_500_5.pdf}
\caption{$\theta=0.50$, $n=500$, $R=5$}
\end{center}
\end{figure}

\begin{figure}
\begin{center}
\includegraphics[width=5.5in, height=2.45in]{../extremal_comparison/figs/sim_frechet_hier_50_1000_5.pdf}
\caption{$\theta=0.50$, $n=1000$, $R=5$}
\end{center}
\end{figure}

\newpage

\begin{figure}
\begin{center}
\includegraphics[width=5.5in, height=2.45in]{../extremal_comparison/figs/sim_frechet_hier_50_250_10.pdf}
\caption{$\theta=0.50$, $n=250$, $R=10$}
\end{center}
\end{figure}

\begin{figure}
\begin{center}
\includegraphics[width=5.5in, height=2.45in]{../extremal_comparison/figs/sim_frechet_hier_50_500_10.pdf}
\caption{$\theta=0.50$, $n=500$, $R=10$}
\end{center}
\end{figure}

\begin{figure}
\begin{center}
\includegraphics[width=5.5in, height=2.45in]{../extremal_comparison/figs/sim_frechet_hier_50_1000_10.pdf}
\caption{$\theta=0.50$, $n=1000$, $R=10$}
\end{center}
\end{figure}

\newpage

\begin{figure}
\begin{center}
\includegraphics[width=5.5in, height=2.45in]{../extremal_comparison/figs/sim_frechet_hier_50_250_20.pdf}
\caption{$\theta=0.50$, $n=250$, $R=20$}
\end{center}
\end{figure}

\begin{figure}
\begin{center}
\includegraphics[width=5.5in, height=2.45in]{../extremal_comparison/figs/sim_frechet_hier_50_500_20.pdf}
\caption{$\theta=0.50$, $n=500$, $R=20$}
\end{center}
\end{figure}

\begin{figure}
\begin{center}
\includegraphics[width=5.5in, height=2.45in]{../extremal_comparison/figs/sim_frechet_hier_50_1000_20.pdf}
\caption{$\theta=0.50$, $n=1000$, $R=20$}
\end{center}
\end{figure}





\newpage

\begin{figure}
\begin{center}
\includegraphics[width=5.5in, height=2.45in]{../extremal_comparison/figs/sim_frechet_hier_85_250_5.pdf}
\caption{$\theta=0.85$, $n=250$, $R=5$}
\end{center}
\end{figure}

\begin{figure}
\begin{center}
\includegraphics[width=5.5in, height=2.45in]{../extremal_comparison/figs/sim_frechet_hier_85_500_5.pdf}
\caption{$\theta=0.85$, $n=500$, $R=5$}
\end{center}
\end{figure}

\begin{figure}
\begin{center}
\includegraphics[width=5.5in, height=2.45in]{../extremal_comparison/figs/sim_frechet_hier_85_1000_5.pdf}
\caption{$\theta=0.85$, $n=1000$, $R=5$}
\end{center}
\end{figure}

\newpage

\begin{figure}
\begin{center}
\includegraphics[width=5.5in, height=2.45in]{../extremal_comparison/figs/sim_frechet_hier_85_250_10.pdf}
\caption{$\theta=0.85$, $n=250$, $R=10$}
\end{center}
\end{figure}

\begin{figure}
\begin{center}
\includegraphics[width=5.5in, height=2.45in]{../extremal_comparison/figs/sim_frechet_hier_85_500_10.pdf}
\caption{$\theta=0.85$, $n=500$, $R=10$}
\end{center}
\end{figure}

\begin{figure}
\begin{center}
\includegraphics[width=5.5in, height=2.45in]{../extremal_comparison/figs/sim_frechet_hier_85_1000_10.pdf}
\caption{$\theta=0.85$, $n=1000$, $R=10$}
\end{center}
\end{figure}

\newpage

\begin{figure}
\begin{center}
\includegraphics[width=5.5in, height=2.45in]{../extremal_comparison/figs/sim_frechet_hier_85_250_20.pdf}
\caption{$\theta=0.85$, $n=250$, $R=20$}
\end{center}
\end{figure}

\begin{figure}
\begin{center}
\includegraphics[width=5.5in, height=2.45in]{../extremal_comparison/figs/sim_frechet_hier_85_500_20.pdf}
\caption{$\theta=0.85$, $n=500$, $R=20$}
\end{center}
\end{figure}

\begin{figure}
\begin{center}
\includegraphics[width=5.5in, height=2.45in]{../extremal_comparison/figs/sim_frechet_hier_85_1000_20.pdf}
\caption{$\theta=0.85$, $n=1000$, $R=20$}
\end{center}
\end{figure}


\end{document}
